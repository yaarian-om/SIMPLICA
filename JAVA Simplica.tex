\documentclass[A4 paper,openany]{book}  % Openany removes the blank pages between 2 chapters
\usepackage{color}
\usepackage{xcolor}
\usepackage{graphicx}                   %  For Inserting List of Figures
\usepackage{index}                      %  For Inserting Index in the document
\usepackage{url,hyperref}               %  For Inserting Index in the document
\usepackage[xindy]{imakeidx}            %  For Inserting Index in Alphabetical order

\makeindex                              %  For Inserting Index in the document

\definecolor{smokeWhite}{HTML}{F5F5F5}
\begin{document}
\begin{titlepage}
    \pagecolor{black}
    \color{white}
    \noindent{\Huge \textbf{JAVA Simplica}}\\
    {\large{$1^{st}$ Edition,}}
    {\Large{Sudipta Kumar Das}}
    \tiny{
        \begin{verbatim}
                                                                                                              
                                                                                                                                                                                                                                                          
              
            























                                                                                                                                 
                                                                                                                                 
                    JJJJJJJJJJJ                    AAA                    VVVVVVVV           VVVVVVVV                    AAA               
                    J:::::::::J                   A:::A                   V::::::V           V::::::V                   A:::A              
                    J:::::::::J                  A:::::A                  V::::::V           V::::::V                  A:::::A             
                    JJ:::::::JJ                 A:::::::A                 V::::::V           V::::::V                 A:::::::A            
                      J:::::J                  A:::::::::A                 V:::::V           V:::::V                 A:::::::::A           
                      J:::::J                 A:::::A:::::A                 V:::::V         V:::::V                 A:::::A:::::A          
                      J:::::J                A:::::A A:::::A                 V:::::V       V:::::V                 A:::::A A:::::A         
                      J:::::j               A:::::A   A:::::A                 V:::::V     V:::::V                 A:::::A   A:::::A        
                      J:::::J              A:::::A     A:::::A                 V:::::V   V:::::V                 A:::::A     A:::::A       
          JJJJJJJ     J:::::J             A:::::AAAAAAAAA:::::A                 V:::::V V:::::V                 A:::::AAAAAAAAA:::::A      
          J:::::J     J:::::J            A:::::::::::::::::::::A                 V:::::V:::::V                 A:::::::::::::::::::::A     
          J::::::J   J::::::J           A:::::AAAAAAAAAAAAA:::::A                 V:::::::::V                 A:::::AAAAAAAAAAAAA:::::A    
          J:::::::JJJ:::::::J          A:::::A             A:::::A                 V:::::::V                 A:::::A             A:::::A   
           JJ:::::::::::::JJ          A:::::A               A:::::A                 V:::::V                 A:::::A               A:::::A  
             JJ:::::::::JJ           A:::::A                 A:::::A                 V:::V                 A:::::A                 A:::::A 
               JJJJJJJJJ            AAAAAAA                   AAAAAAA                 VVV                 AAAAAAA                   AAAAAAA
                                                                                                                                   
                                                                                                                                   
                                                                                                                                   
                                                                                                                                   
                                                                                                                                   
                                                                                                                                   
                                                                                                                                                                                                                                                                        
                                                                                                                                                                                                                                                              
                                                                                                                                                                                                                                                              
                                          _____   _____   __  __   _____    _        _____    _____            
                                         / ____| |_   _| |  \/  | |  __ \  | |      |_   _|  / ____|     /\    
                                        | (___     | |   | \  / | | |__) | | |        | |   | |         /  \   
                                         \___ \    | |   | |\/| | |  ___/  | |        | |   | |        / /\ \  
                                         ____) |  _| |_  | |  | | | |      | |____   _| |_  | |____   / ____ \ 
                                        |_____/  |_____| |_|  |_| |_|      |______| |_____|  \_____| /_/    \_\
                                                                                            
                                                                                                                                                                                                              
                                                                                                                                                              
                                                                                                                                                                                                                                                                     
                                                                                                                                                                                                                                                                            
                                                                                                      
                                                                                                      
        \end{verbatim}
    }
    \vfill
    {\small{2022, Publisher}}
\end{titlepage}


\pagecolor{smokeWhite}
\color{black}
\newpage
\tableofcontents
\newpage
\listoffigures
\newpage
\listoftables
\newpage

% 
% Part 1
% 
\part{Introduction}
JAVA\cite{Ref1} is a Programming language which is used mostly in offial softwares because of it's strong security system.
It is a high-level language which uses JVM to convert the high-level code to a machine code.
It is one of the most popular programming languages out there. Released in 1995 and still widely used today.
Java has many applications, including software development, mobile applications, and large systems development.
Knowing Java opens a lot of possibilities for us as a developer.

\chapter*{Preface}
JAVA knowledge is vast. People most often have to go through most of the documentations of the code then they could think of writing something.
Moreover, sometimes people looses their interest in learning JAVA or write their codes in JAVA. So in that case they just give online posts
and hire outsourcers to complete there school/college projects howmeworks and others.
This processs is both insecure and costly. In this book I just tried to teach JAVA  in a simple way and by which
people can start doing their school/college projects howmeworks and others by their own having simple knowledge. Thus, they can learn the vast knowledge slowly and more interesting way.

\addcontentsline{toc}{chapter}{Preface}
% 
% Chapter 1
% 
\chapter{History of JAVA}
Java was originally developed by James Gosling\cite{Ref2} \index{James Gosling} at Sun Microsystems and released in May 1995 as a core component of Sun Microsystems' Java platform.
The original and reference implementation Java compilers\index{Java compilers}, virtual machines\index{virtual machines}, and class libraries\index{class libraries} were originally released by Sun under proprietary licenses.
As of May 2007, in compliance with the specifications of the Java Community Process, Sun had relicensed most of its Java technologies\index{Java technologies} under the GPL-2.0-only license.
Oracle offers its own HotSpot Java Virtual Machine, however the official reference implementation is the OpenJDK\index{OpenJDK} JVM\index{JVM} which is free open-source software and used by most developers
and is the default JVM for almost all Linux distributions\index{Linux distributions}.

% 
% Figure 1 : James Gosling
% 

\begin{figure}[htbp]
    \begin{center}
        \fbox{\includegraphics*[width=8cm]{JamesGosling.jpg}}
        \caption{James Gosling}
    \end{center}
\end{figure}


% 
% Part 2
% 
\part{Pre-Basic of JAVA}

JAVA is a vast programming language, but it has some pre basic things, on whichs the whole language depends on. In this part we'll going to discuss It

% 
% Chapter 2
% 
\chapter{Package \& Class Declaration}

% 
% Section 2.1
% 
\section{package}
Package is kind of a folder, where all the class files are present. we can use them by using the keyword \textit{import packageName.subPackageName.className} or
\textit{import packageName.*}. Here * means all the things. we can use predefined packages of jdk or we can also import out own packages in any class from another folder.
% 
% Subsection 2.1.1
% 
\subsection{Syntax}
\begin{center}
    \tt{
        \textit{import packageName.subPackageName.className}
    }
\end{center}
% 
% Subsection 2.1.2
% 
\subsection{Example}
\begin{center}
    \begin{verbatim}
        java.io.File;
    \end{verbatim}
\end{center}

% 
% Section 2.2
% 
\section{Access modifiers}
Access\index{Access} modifires basically used to control the access of the variables \& methods form another class or package. It is mostly used in Encapsulation\index{Encapsulation}.
There are basically 4 Access modifiers. Those are,
\begin{itemize}
    \item Public
    \item Private
    \item Protected
    \item Default
\end{itemize}
% 
% Subsection 2.2.1
%
\subsection{Public}
Public\index{Public} Keyword is used to make the variables and methods Public that means those thing can be access from anywhere, no matter where it is.
% 
% Subsection 2.2.2
%
\subsection{Private}
Private\index{Private} Keyword is used to make the variables and methods inaccessible that means those thing can be access from nowhere, no matter where it is.
% 
% Subsection 2.2.3
%
\subsection{Protected}
Protected Keyword is used to make the variables and methods only accessible from their children that means those thing can be access from nowhere except its child class,
no matter where it is. IF a class is extended by another class then the class who extend in it, called child\index{Child Class} class of the class who got extended by the child class.
And that class who got extended by the child class called parent\index{Parent Class} Class.
% 
% Subsection 2.2.3
%
\subsection{Default}
We don't need any access modifiers to make it default access. Default access is kind of private access modifier. Default\index{Default} access means that variable/methods can be accessible
from anywhere inside the folder its in. And can not be accessible outside of the folder.
% 
% Subsection 2.2.4
%
\subsection{Syntax}
\begin{center}
    \tt{
        \textit{Access\_modifier dataType/returnType variableName/methodName()}
    }
\end{center}
% 
% Subsection 2.2.5
% 
\subsection{Example}
\begin{center}
    \begin{verbatim}
        public boolean isAccessible = true;
        private String name = "Sudipta Kumar Das";
        protected String carModel = "Toyota CHR";
        int age = 22; \\ This is Default Access Modifier
    \end{verbatim}
\end{center}
% 
% Section 2.3
% 
\section{Class Declaration}
JAVA is an Object Oriented Programming(OOP) Language. Here we have to use lots of classes. To use classes we have to declare it. Class declaration has its own syntax
% 
% Subsection 2.3.1
% 
\subsection{Syntax}
\begin{center}
    \tt{
        \textit{Access\_modifier class className}
    }
\end{center}
% 
% Subsection 2.3.2
% 
\subsection{Example}
\begin{center}
    \begin{verbatim}
        public class Mobile{

        }
    \end{verbatim}
\end{center}

% 
% Section 2.4
% 
\section{Main Method}
JAVA is a high level language. It needs a compliler to convert the high level code into machione code. The compilers need to understand the starting point of the code conversion.
Main method\index{Main Method} is the place from where the compilers start reading and start compiling. There should be only one main method for entire program or project. Classes can be many but main method
must be one. main method is declared inside any one class.
% 
% Subsection 2.4.1
% 
\subsection{Syntax}
\begin{center}
    \tt{
        \textit{public static void main(String[] args)\{\}}
    }
\end{center}
% 
% Subsection 2.4.2
% 
\subsection{Example}
\begin{center}
    \begin{verbatim}
        public class Test{
            public static void main(String[] args){

            }
        }
    \end{verbatim}
\end{center}

% 
% Section 2.5
% 
\section{Show Output in JAVA}
We use \textit{System.out.println();} to print anything or show anything on console. Here println means print a newline also.
That means the line will break and go to a new line after showing the output inside first bracket.
% 
% Subsection 2.5.1
% 
\subsection{Syntax}
\begin{center}
    \tt{
        \textit{System.out.println();}
    }
\end{center}
% 
% Subsection 2.5.2
% 
\subsection{Example}
\begin{center}
    \begin{verbatim}
        public class Test{
            public static void main(String[] args){
                System.out.println("HELLO WORLD !");
            }
        }
    \end{verbatim}
\end{center}
% 
% Figure 2 : Show Output in JAVA
% 
\begin{figure}[htbp]
    \begin{center}
        \fbox{\includegraphics*[width=8cm]{helloWorld.png}}
        \caption{Show Output in JAVA\cite{Ref3}}
    \end{center}
\end{figure}

% 
% Chapter 3
% 
\chapter{Escape Sequence \& Format Specifier}
% 
% Section 3.1
%
\section{Escape Sequence}
Escape sequences\cite{Ref4}\index{Escape Sequences} are some special characters who perfoms some special kinds of works on showing console output as like printing a backslash or a new line.
Escape sequences are written after a backslash indicating it is a special character. And it is been written inside double quote marks("").
\begin{table}[htbp]
    \begin{tabular}{cl}
        Escape Sequence                  & \multicolumn{1}{c}{Meaning}                                                                                        \\
        \textbackslash{}b                & Backspace                                                                                                          \\
        \textbackslash{}t                & Tab (4 spaces at right)                                                                                            \\
        \textbackslash{}n                & New Line/Break Line                                                                                                \\
        \textbackslash{}r                & \begin{tabular}[c]{@{}l@{}}Carriage Return/\\ Break line \& tart from the left most\\ after this line\end{tabular} \\
        \textbackslash{}"                & Print Double quote mark on console                                                                                 \\
        \textbackslash{}'                & Print Single quote mark on console                                                                                 \\
        \textbackslash{}f                & Insert a form feed in the text at this point.                                                                      \\
        \textbackslash{}\textbackslash{} & Print Backslash on console
    \end{tabular}
    \centering
    \caption{Escape Sequences}
\end{table}
% 
% Subsection 3.1.1
%
\subsection{Syntax}
\begin{center}
    \tt{
        \textit{"\textbackslash escapeCharacter"}
    }
\end{center}
% 
% Subsection 3.1.2
%
\subsection{Example}
\begin{center}
    \begin{verbatim}
        public class Test {
            public static void main(String args[]) {
                System.out.println("HELLO\b WORLD !");
                System.out.println("HELLO\t WORLD !");
                System.out.println("HELLO\n WORLD !");
                System.out.println("HELLO\r WORLD !");
                System.out.println("HELLO \"WORLD\" !");
                System.out.println("HELLO \`W\'ORLD !");
                System.out.println("HELLO\f WORLD !");
                System.out.println("HELLO \\WORLD !");
            }
        }
    \end{verbatim}
\end{center}
% 
% Figure 3 : Escape Sequence in java
% 
\begin{figure}[htbp]
    \begin{center}
        \quad\quad\quad\quad\fbox{\includegraphics*[width=8cm]{escapeSequence.png}}
        \caption{Escape Sequences\cite{Ref4}\cite{Ref3}}
    \end{center}
\end{figure}
% 
% Section 3.2
%
\section{Format Specifier}
Format Specifier\cite{Ref5} is used to indicate the place where the value of a variable should appear in a string. That means sometimes we have to show the output inside a line, as like.
Hii! I am \{age\} year old. here we want age = 22 or something just like that. So we'll write  System.out.println("Hii! I am \%d year old.",age); Here output will be if age = 22,
Hii! I am 22 year old.

\begin{table}[htbp]
    \begin{tabular}{ccl}
        Format Specifier & Usual Variable Type & \multicolumn{1}{c}{Display As}                                                                              \\
        \%f\%f           & float or double     & Signed Decimal                                                                                              \\
        \%o              & int                 & unsigned Octal value                                                                                        \\
        \%u              & int                 & unsigned Integer                                                                                            \\
        \%x              & int                 & unsigned Hex value                                                                                          \\
        \%H              & int                 & unsigned Decimal Integer                                                                                    \\
        \%S              & array of char       & Sequence of Characters                                                                                      \\
        \%\%             & -                   & Inserts a \% sign                                                                                           \\
        \%f              & float               & Decimal floating-point                                                                                      \\
        \%e\%E           & -                   & \begin{tabular}[c]{@{}l@{}}Scientific Notation /\\ Exponential Format\end{tabular}                          \\
        \%g              & -                   & \begin{tabular}[c]{@{}l@{}}Causes formatter to use \\ either \%f or \%e which one\\ is shorter\end{tabular} \\
        \%h\%H           & -                   & Hash code of the Argument                                                                                   \\
        \%d              &                     & Decimal Integer                                                                                             \\
        \%c              & \textbf{}           & Character                                                                                                   \\
        \%b\%B           & boolean             & Boolean                                                                                                     \\
        \%a\%A           & -                   & Floating Point hexadecimal
    \end{tabular}
    \centering
    \caption{Format Specifier}
\end{table}
% 
% Subsection 3.2.1
%
\subsection{Syntax}
\begin{center}
    \tt{
        \textit{"\%formatSpecifier"}
    }
\end{center}
% 
% Subsection 3.2.2
%
\subsection{Example}
\begin{center}
    \begin{verbatim}
        public class Test {
            public static void main(String args[]) {
                int i = 1234567890;
                boolean b = true;
                char c = 'a';
                short s = 12345;
                float f = 10.2f;
                double d = 344.659;
                System.out.printf("boolean b = %b\n",b);
                System.out.printf("charater c = %c\n",c);
                System.out.printf("short s = %d\n",s);
                System.out.printf("integer i = %d\n",i);
                System.out.printf("float f = %1f\n",f);
                System.out.printf("double d = %3f\n",d);
            }
        }
    \end{verbatim}
\end{center}
% 
% Figure 3 :Format Specifier in java
% 
\begin{figure}[htbp]
    \begin{center}
        \fbox{\includegraphics*[width=8cm]{formatSpecifier.png}}
        \caption{Format Specifier\cite{Ref5}\cite{Ref3}}
    \end{center}
\end{figure}

















% 
% INDEX
% 
\printindex
% 
% BIBLIOGRAPHY
% 
\begin{thebibliography}{90}
    \bibitem{Ref1} Java, \url{https://en.wikipedia.org/wiki/Java_(programming_language)}
    \bibitem{Ref2} James Gosling, \url{https://en.wikipedia.org/wiki/James_Gosling}
    \bibitem{Ref3} Java Compiler, \url{https://www.jdoodle.com/online-java-compiler/}
    \bibitem{Ref4} Escape Sequence, \url{https://docs.oracle.com/javase/tutorial/java/data/characters.html}
    \bibitem{Ref5} Specifier, \url{https://www.geeksforgeeks.org/format-specifiers-in-java/}
\end{thebibliography}

\end{document}

% \index{Index B}
% \chapter{Chapter 2 of Part 2}
% \index{Index C}
% \addtocontents{toc}{DESCRIPTION} % Used for writing a DESCRIPTION of any chapter/section, It means the name will added in contents but page no does not appear
% \section{Section of Chapter 2 of part 2}

% 
% Figure 0 : Temp
% 
\begin{figure}[htbp]
    \begin{center}
        \fbox{\includegraphics*[width=8cm]{temp.jpg}}
        \caption{temp\cite{Ref3}}
    \end{center}
\end{figure}

% 
% Part 1
% 
\index{Private}

\cite{Ref1}